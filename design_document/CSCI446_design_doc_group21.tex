%%%%%%%%%%%%%%%%%%%%%%%%%%%%%%%%%%%%%%%%%
% Journal Article
% LaTeX Template
% Version 1.3 (9/9/13)
%
% This template has been downloaded from:
% http://www.LaTeXTemplates.com
%
% Original author:
% Frits Wenneker (http://www.howtotex.com)
%
% License:
% CC BY-NC-SA 3.0 (http://creativecommons.org/licenses/by-nc-sa/3.0/)
%
%%%%%%%%%%%%%%%%%%%%%%%%%%%%%%%%%%%%%%%%%

%----------------------------------------------------------------------------------------
%	PACKAGES AND OTHER DOCUMENT CONFIGURATIONS
%----------------------------------------------------------------------------------------

\documentclass[twoside]{article}

%\documentclass{aastex}  % version 5.0 or prior
%\usepackage{natbib}



\usepackage{graphicx}
\usepackage{lipsum} % Package to generate dummy text throughout this template
\usepackage[sc]{mathpazo} % Use the Palatino font
\usepackage[T1]{fontenc} % Use 8-bit encoding that has 256 glyphs
\linespread{1.05} % Line spacing - Palatino needs more space between lines
\usepackage{microtype} % Slightly tweak font spacing for aesthetics

\usepackage[margin=1in,columnsep=20pt]{geometry} % Document margins
\usepackage{multicol} % Used for the two-column layout of the document
\usepackage[hang, small,labelfont=bf,up,textfont=it,up]{caption} % Custom captions under/above floats in tables or figures
\usepackage{booktabs} % Horizontal rules in tables
\usepackage{float} % Required for tables and figures in the multi-column environment - they need to be placed in specific locations with the [H] (e.g. \begin{table}[H])
\usepackage{hyperref} % For hyperlinks in the PDF
\usepackage{subcaption}

\usepackage{lettrine} % The lettrine is the first enlarged letter at the beginning of the text
\usepackage{paralist} % Used for the compactitem environment which makes bullet points with less space between them
\usepackage{amsmath}
\usepackage{abstract} % Allows abstract customization
\renewcommand{\abstractnamefont}{\normalfont\bfseries} % Set the "Abstract" text to bold
\renewcommand{\abstracttextfont}{\normalfont\small\itshape} % Set the abstract itself to small italic text

\usepackage{titlesec} % Allows customization of titles
%\renewcommand\thesection{\Roman{section}} % Roman numerals for the sections
%\renewcommand\thesubsection{\Roman{subsection}} % Roman numerals for subsections
%\renewcommand\thesubsubsection{\Alph{subsubsection}} % Roman numerals for subsections
\titleformat{\section}[block]{\Large\scshape}{\thesection}{1em}{} % Change the look of the section titles
\titleformat{\subsection}[block]{\large}{\thesubsection}{1em}{} % Change the look of the section titles
\titleformat{\subsubsection}[block]{}{\thesubsubsection}{1em}{} % Change the look of the section titles

\usepackage{fancyhdr} % Headers and footers
\pagestyle{fancy} % All pages have headers and footers
\fancyhead{} % Blank out the default header
\fancyfoot{} % Blank out the default footer
\fancyhead[C]{CSCI 466 Artificial Intelligence \quad $\bullet$ \quad \today} % Custom header text
\fancyfoot[RO,LE]{\thepage} % Custom footer text


%----------------------------------------------------------------------------------------
%	TITLE SECTION
%----------------------------------------------------------------------------------------

\title{\vspace{-15mm}\fontsize{24pt}{10pt}\selectfont\textbf{CSCI 446 Artificial Intelligence \\ Project 1 Design Report} \\[4mm] \Large Montana State University} % Article title
\date{}
\author{ 
\large
\textsc{Roy Smart} \and \textsc{Nevin Leh} \and \textsc{Brian Marsh} \\[2mm] % Your name
}


%----------------------------------------------------------------------------------------

\begin{document}

\maketitle % Insert title

\thispagestyle{fancy} % All pages have headers and footers

%\begin{abstract}
%We present a novel way of performing MOSES data inversions using a
%\end{abstract}

%----------------------------------------------------------------------------------------
%	ARTICLE CONTENTS
%----------------------------------------------------------------------------------------

\begin{multicols}{2} % Two-column layout throughout the main article text

\section{Introduction}
The \textit{Graph Coloring Problem} (GCP) is the problem of attempting to color a set of bordering regions such that no region has the same color as its neighbors using three or four colors. For example consider the problem of coloring a map of the USA (Figure \ref{usa}), using only four colors and ensuring that no neighboring states share the same color. This is the motivation of the graph coloring problem. 
\begin{figure}[H]
	\centering
	\includegraphics[width=\linewidth]{../images/usa}
	\caption{Map of the United States of America satisfying the graph coloring problem.}
	\label{usa}
\end{figure}
It can be shown that the map coloring problem reduces to the graph coloring problem if we represent the states as the vertices of the graph, and the borders between states as the edges of the graph. This configuration produces a \textit{maximally planar} graph, a graph with no edge intersections and adding any edge would result in an edge intersection.
We are tasked with solving this problem five different ways: Minimum Conflicts, Simple Backtracking, Backtracking with Forward Checking, Backtracking with Constraint Propagation (MAC), and Local Search using a Genetic Algorithm. To test these algorithms, we will first need to build a problem generating program that can produce a random set of \textit{maximally planar} graphs. Using the problem generator, we will calculate a set of graphs between the sizes {10,20,30,...,100} and then use the five graph coloring algorithms to solve the GCP. We will measure GCP algorithm performance by how many vertex colorings it requires to find a solution.
\section{Problem Generation}
The graph coloring problem
\section{Experiment Design}

\section{Solution Overview}

\section{Graph Coloring Algorithms}
\subsection{Minimum Conflicts}
\subsection{Simple Backtracking}
\subsection{Backtracking with Forward Checking}
\subsection{Backtracking with Constraint Propagation}
\subsection{Local Search using a Genetic Algorithm}


\end{multicols}

	%\bibliographystyle{apj}
%	\bibliographystyle{unsrt}
%	\bibliography{sources}
\end{document}